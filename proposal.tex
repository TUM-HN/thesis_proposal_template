\RequirePackage[l2tabu,orthodox]{nag}

% TODO: decide if one-sided/two-side
%\documentclass[headsepline,footsepline,footinclude=false,fontsize=11pt,paper=a4,listof=totoc,bibliography=totoc,BCOR=12mm,DIV=12]{scrbook} % two-sided
\documentclass[headsepline,footsepline,footinclude=false,oneside,fontsize=11pt,paper=a4,listof=totoc,bibliography=totoc]{scrbook} % one-sided

\input{settings}

% TODO: change thesis information
\newcommand*{\getUniversity}{Technical University of Munich}
\newcommand*{\getFaculty}{School of Computation, Information and Technology }
\newcommand*{\getTitle}{<<Thesis title>>}
\newcommand*{\getTitleGer}{<<Titel der Abschlussarbeit>>}
\newcommand*{\getAuthor}{<<Student's name>>}
\newcommand*{\getDoctype}{<<Bachelor's / Master's>> Thesis in Information Engineering \ldots}
\newcommand*{\getSupervisor}{Prof. Dr. Stefan Wagner}
\newcommand*{\getAdvisor}{Dr. Santiago Berrezueta}
\newcommand*{\getSubmissionDate}{<<Submission date>>}
\newcommand*{\getSubmissionLocation}{Heilbronn}

\begin{document}

\pagenumbering{alph}
\frontmatter{}
\input{pages/title}
\mainmatter{}

\chapter{Abstract}
% The Abstract provides a brief overview of the proposal, summarizing the key elements in a concise way. It should:
% - Include a short description of the main problem or research question.
% - Provide the motivation behind choosing this topic.
% - Outline the objectives and goals.
% - Summarize the proposed approach or methods.
% - Mention the expected results or outcomes.
% The abstract should be brief (typically around 150-200 words) and give the reader a quick understanding of the purpose and direction of the research.

\chapter{Thesis Proposal}

\section{Introduction}
% The Introduction introduces the research topic and provides context. It should:
% - Define the research area and provide background information.
% - Explain the importance and relevance of the research topic.
% - Present a high-level overview of the problem or research gap that the study addresses.
% - End with a brief outline of the structure of the proposal.
% This section sets the stage for the supervisor(s), helping them understand why the research is important and what the proposal aims to achieve.

\section{Problem}
% The Problem section clearly defines the specific issue, question, or gap in knowledge that the research aims to address. It should:
% - Describe the nature and scope of the problem.
% - Highlight the aspects that make it a significant issue worth studying.
% - Identify any limitations in current approaches or gaps in existing knowledge that the study seeks to fill.
% This section is crucial as it justifies the need for the study by clearly outlining what is problematic or unknown.

\section{Motivation}
% The Motivation section provides the reasons and background for pursuing this particular research. It should:
% - Explain why this problem or question is personally, scientifically, or socially relevant.
% - Discuss any specific factors that led the researcher to choose this topic.
% - Highlight the potential impact of the research findings on the field or in a practical context.
% This section helps the reader understand the researcher’s interest in the topic and its broader significance.

\section{Objectives}
% The Objectives section outlines the specific goals and intended outcomes of the research. It should:
% - List the primary objectives of the study, usually as bullet points.
% - Differentiate between primary and secondary objectives if necessary.
% - Provide clear and measurable targets that the research will aim to achieve.
% This section serves as a roadmap for the study, clarifying what the research is aiming to accomplish.

\section{Schedule}

% The Schedule provides a timeline for completing the research, showing when each key phase or activity will be carried out. It should:
% - Break down the research process into stages (e.g., literature review, data collection, analysis, writing).
% - Include estimated start and end dates for each stage.
% - Optionally, present this information in a table or Gantt chart for clarity.
% A well-defined schedule is essential in demonstrating that the project is feasible within the allotted time and resources.

\printbibliography{}

% The bibliography section must contain at least 15 references ONLY from books, scientific peer-reviewed papers, Doctoral Thesis, and/or previous BSc/MSc thesis. 


% - Ensure that all cited sources are relevant and contribute meaningfully to your argument or background.

% - Use a consistent citation style throughout the thesis (ideally alpha with [ABC12]).

% - Only include peer-reviewed conference papers or journal articles in your bibliography.
% - Avoid including internet sources in the literature; if used at all, include them as footnotes.
% - Clean up your bibliography to avoid duplicate or incorrect information (e.g., location details for ACM conferences).
% - Avoid simply copying and pasting Google Scholar entries; manual cleanup is often required.
% - The citation should be placed before the full stop (e.g., some example text [AB12].) and not after the full stop.
% - Regularly cross-check in-text citations with the bibliography to ensure all sources are listed and correctly referenced.

%If citing textbooks or technical reports, ensure they are authoritative and well-established within the field.

\end{document}
